\documentclass[a4paper]{article}

\usepackage{import}
\usepackage{xcolor}
\usepackage{textcomp,mathcomp}
\usepackage{cancel}
\usepackage{float}
\usepackage{forloop}
\usepackage{pgfplots}
\usepackage{booktabs}
\usepgfplotslibrary{fillbetween}


\usepackage[version=4]{mhchem}
\usepackage{multido}

\definecolor{yorhabg}{HTML}{C8C2AA}
\definecolor{yorhafg}{HTML}{4D493E}
\definecolor{yorhagrid}{HTML}{B5AF9C}
\definecolor{gruvred}{HTML}{91221D}

\pagecolor{yorhabg}
\color{yorhafg}

% \import{project1 (pblock)}{preamble.sty}

% Removes padding above title
\usepackage{titling}
\setlength{\droptitle}{-10em}

% Font package
\usepackage[T1]{fontenc}

\usepackage{fouriernc}

\usepackage{calc}


\usepackage{sectsty}
\usepackage{graphicx}
\usepackage{amsmath}
\usepackage{amssymb}
\usepackage[most]{tcolorbox}

\usepackage{tikz}
\usepackage{eso-pic}
\usetikzlibrary{calc,shadows.blur}

\usepackage[backend=biber,style=ieee,citestyle=numeric-comp,sorting=none]{biblatex}
\addbibresource{sample.bib}

\AddToShipoutPictureBG{%
	\begin{tikzpicture}[remember picture, overlay,
		help lines/.append style={line width=0.05pt, color=yorhagrid}]
		\draw[help lines] (current page.south west) grid[step=5pt]
		(current page.north east);
	\end{tikzpicture}%
}

% Margins
\topmargin=0in
\evensidemargin=0in
\oddsidemargin=0in
\textwidth=6.5in
\textheight=9.0in
\headsep=0.25in

\AtBeginEnvironment{tcolorbox}{\small}

\newtcolorbox[auto counter, number within=section]{question}{%
	enhanced,
	colback=yorhabg,
	colframe=yorhafg,
	coltext=yorhafg,
	coltitle=yorhabg,
	title={\textbf{Question~\thetcbcounter}},
	arc=0pt,
	outer arc=0pt,
	drop shadow southeast,
	sharp corners,
}

\newtcolorbox{solution}{%
	enhanced,
	breakable,
	colback=yorhabg,
	colframe=yorhafg,
	coltext=yorhafg,
	coltitle=yorhabg,
	arc=0pt,
	outer arc=0pt,
	drop shadow southeast,
	sharp corners,
	before skip = 0pt,
}

\newtcolorbox{imp}{enhanced,arc=0mm,colback=yorhabg,colframe=yorhafg,leftrule=10mm,coltext=yorhafg,%
	overlay={\node[anchor=west,outer sep=1pt] at (frame.west) {\scalebox{1}{\textcolor{yorhabg}{\textbf{Info}}}}; }}

\newcommand\bb[1]{\textcolor{yorhafg}{\textbf{#1}}}

\newcommand\ph{\textrm{pH}}
\newcommand\poh{\textrm{pOH}}

\newcommand\mybox[2][]{\tikz[overlay]\node[fill=yorhafg, color=yorhabg, inner sep=2pt, anchor=text, rectangle, rounded corners=1mm,#1] {#2};\phantom{#2}}


\pgfplotsset{
	every axis/.append style={
		legend style={
			fill=yorhagrid,     % dark background
			draw=yorhafg,        % white border
			text=yorhafg,
			rounded corners
		}
	}
}


\title{Math Prep Course Day One}
\author{Ashhad}

\begin{document}

\maketitle

\section{Logic Exercises}
\begin{question}{}{}
Truth table for \(P \land (Q \lor R) \iff (P \land Q) \lor (P \land R)\)
\end{question}

Table for the left hand side

\[
\begin{array}{|c|c|c|c|c|}
\midrule
P & Q & R & Q \lor R & P \land (Q \lor R) \\
\midrule
T & T & T & T & T \\
T & T & F & T & T \\
T & F & T & T & T \\
T & F & F & F & F \\
F & T & T & T & F \\
F & T & F & T & F \\
F & F & T & T & F \\
F & F & F & F & F \\
\midrule
\end{array}
\]

Table for the Right hand side

\[
\begin{array}{|c|c|c|c|c|c|}
\midrule
P & Q & R & P \land Q & P \land R  & (P \land Q) \lor (P \land R) \\
\midrule
T & T & T & T & T & T \\
T & T & F & T & F & T \\
T & F & T & F & T & T \\
T & F & F & F & F & F \\
F & T & T & F & F & F \\
F & T & F & F & F & F \\
F & F & T & F & F & F \\
F & F & F & F & F & F \\
\midrule
\end{array}
\]

Final table
\[
\begin{array}{|c|c|c|}
\midrule
P \land (Q \lor R) &(P \land Q) \lor (P \land R) & P \land (Q \lor R) \iff (P \land Q) \lor (P \land R)\\
\midrule
T & T & T \\
T & T & T \\
T & T & T \\
F & F & T \\
F & F & T \\
F & F & T \\
F & F & T \\
F & F & T \\
\midrule
\end{array}
\]

Thus the two logical statements are equivalent for all truth values.
\begin{question}{}{}
Prove or disprove
\[
\forall x \in \mathbb{R}, \ x^2 > 0
\]
\end{question}

For the universal quantifier \(\forall\), a solution by contradiction is sufficient. We show that there exists an element \(x \in \mathbb{R}\) such that \(x^2 \ngtr 0\). 
\\ \\
Take \(x = 0\), then \(x^2 = 0 \ngtr 0\), and since we have \(\exists x \in \mathbb{R} , \ x^2 \ngtr 0\) then the statement \(\forall x \in \mathbb{R}, \ x^2 > 0\) cannot simultaneously hold, thus contradiction.


\begin{question}{}{}
Prove or disprove
\[
\forall n \in \mathbb{N}, \ n^2 > 2
\]
\end{question}

Following the same pattern as last time, we choose the natural number \(n=1\) and obtain \(n^2 = 1 \ngtr 2\), thus obtaining our contradiction.


\begin{question}{}{} Prove or Disprove
\[
\forall x \in \mathbb{R}, \ \exists y, \ y< x
\]
\end{question}

For any real number \(x \in \mathbb{R}\) we know that \(x-1\) is a well-defined real number. We also know that \(x-1 < x\). Therefore, for each real number \(x\), there exists at least the number \(y = x-1\) such that \(y < x\), thus the proposition holds true.


\section{Set Operations}
\begin{question}{}{}
Take the following sets
\begin{align*}
A &= \{-1, 0, 3, 4, 5, 6\} \\
B &= \{0, 1, 2, 3, 4\} \\
C &= \{0,2,4\} \\ 
\end{align*}
Find
\(A \cap C, \ A \cap B,  \ B \cap C, \ A \cap B \cap C\)
\end{question}

\begin{align*}
A \cap B &= \{0,3,4\} \\
A \cap C &= \{0,4\} \\
B \cap C &= \{0,2,4\} \\
A \cap B \cap C &= \{0, 4\}
\end{align*}

\section{Summation and Product Formulae}
\begin{question}{}{}
Evaluate
\[
\sum_{x=2}^4 (x^2-x)
\]
\end{question}
Use \(x^2 -x = x(x-1)\)
Then our expression becomes via substitution
\begin{align*}
\sum_{x=2}^4 x(x-1) &= 2(2-1) + 3(3-1) + 4(4-1) \\
&= 2 + 6 + 12 \\
&=20
\end{align*}

\begin{question}{}{}
Evaluate
\[
\prod_{x=0}^5 x^2
\]
\end{question}
Since \(0^2 = 0\), the first term \(x = 0\) reduces the entire product to zero. More formally we may write
\begin{align*}
\prod_{x=5}^5 x^2 &= 0^2 \times \prod_{x=1}^5x^2 \\ 
&= 0
\end{align*}

\begin{question}{}{}
Evaluate
\[
\prod_{x=1}^{4}x^2-(x-1)^2
\]
\end{question}
Simplifying \(x^2 - (x-1)^2 = 2x -1\)
\begin{align*}
\prod_{x=1}^4 (2x-1) &= (2-1)(4-1)(6-1)(8-1) \\
&=105
\end{align*}

\begin{question}{}{}
Evaluate
\[
\sum_{x=1}^{100} x
\]
\end{question}
This question is best solved with the aid of a useful and rather famous identity \(\sum_{x=1}^n = \frac{n(n+1)}{2}\). But if you think it is unfair to rely on prior knowledge consider this solution:
\begin{align*}
\sum_{x=1}^{100}x &= \sum_{x=1}^{49}x + \sum_{x=1}^{49}(100-x) + 50 + 100 \\
&= \sum_{x=1}^{49}100 + 150 \\
&= 5050
\end{align*}

\section{Binomial Theorem}
\begin{question}{}{}
Evaluate
\[
(10.1)^3
\]
\end{question}
Expand the term inside the cube to \((10+0.1)^3\)
\begin{align*}
(10+0.1)^3 &= \sum_{i=0}^3 {3 \choose i} (10)^i(0.1)^{3-i} \\
&= \sum_{i=0}^3 {3 \choose i}(10)^i(10^{-1})^{3-i} \\
&= \sum_{i=0}^3 {3 \choose i}(10)^i(10)^{i-3} \\
&= \sum_{i=0}^3 {3 \choose i}(10)^{2i-3} \\
&= (10)^{-3} +3(10)^{-1} + 3(10) + (10)^3
\end{align*}


\begin{question}{}{}
Evaluate
\[
(99)^3
\]
\end{question}
Expand the term inside the cube to \((100+(-1))^3\)
\begin{align*}
(100-1)^3 &= \sum_{i=0}^3 {3 \choose i} (100)^i(-1)^{3-i} \\
&= (-1)^{3} + 3(100)(-1)^2 + 3(100)^2(-1) + (100)^3
\end{align*}

\section{Inequalities}
\begin{question}{}{}
\[
\left| |x+2|-|x-2| \right| > 2
\]
\end{question}
We begin by first splitting the intervals of \(|x+2|\) and \(|x-2|\) respectively to simplify the equation. Thus we obtain
\begin{align*}
|x+2| = 
\begin{cases}
    x+2, &x \geq -2 \\
    -x-2, &x \leq -2
\end{cases}
\end{align*}
Similarly
\begin{align*}
|x-2| = 
\begin{cases}
    x-2, &x \geq 2 \\
    2-x, &x \leq 2
\end{cases}
\end{align*}
Thus our intervals become \(\{(-\infty, -2], (-2, 2), \ [2, \infty)\} \).
Evaluating our functions on these intervals, we obtain
\begin{align*}
(-x-2) - (2-x) &= -4 & x \in (-\infty, -2] \\
(x+2) - (2-x) &= 2x & x \in (-2, 2) \\
(x+2) - (x-2) &= 4 & x \in [2, \infty)
\end{align*}
Notice that for the first and last case, the absolute value is always greater than 2. Therefore we direct our attention to the middle case where we impose the condition
\begin{align*}
|2x| &> 2 && x \in (-2, 2) \\
|x| &>  1
\end{align*}
The solutions to which in the interval \((-2, 2)\) are \((-2, -1)\) and \((1, 2)\).
\\ \\
Our solution set is therefore \((-\infty, -2] \cup (-2, -1) \cup (1, 2) \cup [2, \infty)\). Simplifying, we obtain
\[
(-\infty, -1) \cup (1, \infty)
\]

\begin{center}
	\begin{tikzpicture}
		\begin{axis}[axis lines=middle, ymin=0, ymax=5, xmin=-5, xmax=5, width = 12cm, height =8cm, title={Individual graphs}]
			\addplot[color=red, thick,]{-x-2};
			\addlegendentry{\(|x+2|\)}
			\addplot[color=blue, thick,]{x-2};
			\addlegendentry{\(|x-2|\)}
			\addplot[color=red, thick, dashed, domain=-1:1]{x+2};
			\addplot[color=red, thick, domain=-5:-1]{x+2};
			\addplot[color=red, thick, domain=1:5]{x+2};
			\addplot[color=blue, thick, dashed, domain=-1:1]{2-x};
			\addplot[color=blue, thick, domain=1:5]{2-x};
			\addplot[color=blue, thick, domain=-5:-1]{2-x};
			\addplot[color=yorhafg, dashed] coordinates {(1,0) (1,3)};
			\addplot[color=yorhafg, dashed] coordinates {(-1,0) (-1,3)};
		\end{axis}
	\end{tikzpicture}
\end{center}


The blue line indicates the function \(|x-2|\) whilst the red indicates \(|x+2|\)

\begin{center}
	\begin{tikzpicture}
		\begin{axis}[axis lines=middle, ymin=0, ymax=5, xmin=-5, xmax=5, width = 12cm, height =8cm, title={Graph of \(||x+2|-|x-2||\)}]
			\addplot[color=blue, thick, domain=-5:-2]{4};
			\addplot[color=blue, thick, domain=2:5]{4};
			\addplot[color=red, thick, domain=1:2]{2*x};
			\addplot[color=red, thick, domain=-2:-1]{-2*x};
			\addplot[color=yorhafg, dashed, domain=-1:1]{2*x};
			\addplot[color=yorhafg, dashed, domain=-1:1]{-2*x};
			\addplot[color=yorhafg, dashed] coordinates {(1,0) (1,2)};
			\addplot[color=yorhafg, dashed] coordinates {(-1,0) (-1,2)};
		\end{axis}
	\end{tikzpicture}
\end{center}



\begin{question}{}{}
\[
(x-3)(x+2)x < 0
\]
\end{question}
The roots of the polynomial given are
\begin{align*}
x &= 0 \\
x+2 &= 0 \implies x = -2 \\
x-3 &= 0 \implies x = 3 \\
\end{align*}
Thus our intervals become \(\{(-\infty, -2), (-2, 0), (0, 3), (3, \infty)\}\)\footnote{We exclude the roots themselves since the polynomial is zero at those points and thus not less than zero as the condition requires}
Evaluating our polynomials on the following intervals we observe that
\begin{align*}
(x-3)(x+2)x < 0 &&x \in (-\infty, -2)
\end{align*}
Since all three terms are negative here, this interval satisfies our condition.
\begin{align*}
(x-3)(x+2)x < 0 &&x \in (-2, 0)
\end{align*}
Here only the \(x-3\) and \(x\) terms are negative, thus we have an overall positive evaluation. This interval does not satisfy our condition.
\begin{align*}
(x-3)(x+2)x < 0 &&x \in (0, 3)
\end{align*}
Since only the \(x-3\) term is negative, this interval too satisfies our condition.
\begin{align*}
(x-3)(x+2)x < 0 &&x \in (3, \infty)
\end{align*}
Beyond this point all our vales are positive, not satisfying our condition.
We can sum up our results in the following table (Courtesy of Sherlock for the idea!)

\[
\begin{array}{c|c|c|c|c}
Interval& x-3 & x+2 & x & Product \\
\midrule
(-\infty, -2) &- & - & - & - \\
(-2, 0) &- & + & - & + \\
(0,3) &- & + & + & - \\
(3, \infty) &+ & + & + & + 
\end{array}
\]

Our solution is therefore
\begin{align*}
(-\infty, -2) \cup (0, 3)
\end{align*}

\begin{center}
	\begin{tikzpicture}
		\begin{axis}[axis lines=middle, 
			ymin=-10, ymax=5, xmin=-5, xmax=5, 
			width = 12cm, height =8cm]
			
			\addplot[name path = curve, color=gruvred, thick, samples=50]{(x-3)*(x+2)*x};
			\addlegendentry{\((x-3)(x+2)x\)}
			\addplot[name path = axis, draw=none] {0};
			\addplot[yorhagrid] fill between[of=curve and axis, soft clip = {domain = -5:-2}];
			\addplot[yorhagrid] fill between[of=curve and axis, soft clip = {domain = 0:3}];
		\end{axis}
	\end{tikzpicture}
\end{center}


\begin{question}{}{}
\[
-2 < 1 - \frac{1}{x} < 0
\]
\end{question}
Simplify the expression to \(-2 < \frac{x-1}{x} < 0\)
\begin{align*}
-2x < x -1 < 0 \\
2x > 1-x > 0 \\
\end{align*}
Since the final \(> 0\) condition ensures that \(1-x\) is positive, we can divide the equation by it without having to check for its signature. Therefore
\begin{align*}
\frac{2x}{1-x} > 1 > 0
\end{align*}
The last inequality is trivial so we omit it. We now check when \(\frac{2x}{1-x} > 1\) or rather, \(\frac{3x-1}{1-x} > 0\). Observe that \(1-x\) is positive on the domain \((-\infty, 1)\) and negative otherwise\footnote{We omit the point 1 since our expression is undefined at that point}, similarly \(3x-1\) is positive on the domain \((\frac{1}{3}, \infty)\) and negative otherwise. \\ 

All in all this gives us the three domains \(\{(-\infty, \frac{1}{3}), (\frac{1}{3}, 1), (1, \infty)\}\). We must now evaluate our numerator and denominator on each domain. Again, we can use a handy table to quickly arrange and summarize our results.

\[
\begin{array}{c|c|c|c}
Interval & 3x-1 & 1 - x & Product \\
\midrule
\left(-\infty, \frac{1}{3}\right) & - & + & - \\[3pt]
\left(\frac{1}{3}, 1\right) & + & + & + \\[3pt]
(1, \infty) & + & - & -
\end{array}
\]

The only valid interval is therefore 
\[
\left(\frac{1}{3}, 1\right)
\]

\begin{center}
	\begin{tikzpicture}
		\begin{axis}[axis lines=middle, 
			ymin=-3, ymax=3, xmin=-2, xmax=2, 
			width = 12cm, height =8cm]
			
			\addplot[name path=curve, color=gruvred, thick, samples=50, domain=-2:2]{1-1/x};
			\addlegendentry{\(1 - \frac{1}{x}\)}
			\addplot[color=yorhafg, dashed] coordinates {(1/3,-3) (1/3,3)};
			\addplot[name path=axis, draw=none]{0};
			\addplot[yorhagrid] fill between[of = curve and axis, soft clip = {domain = 1/3:1}];
			
		\end{axis}
	\end{tikzpicture}
\end{center}


\end{document}

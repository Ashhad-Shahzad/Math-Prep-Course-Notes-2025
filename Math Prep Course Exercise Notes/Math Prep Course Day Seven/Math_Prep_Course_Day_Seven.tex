\documentclass[a4paper]{article}

\usepackage{import}
\usepackage{xcolor}
\usepackage{textcomp,mathcomp}
\usepackage{cancel}
\usepackage{float}
\usepackage{forloop}
\usepackage{pgfplots}
\usepackage{booktabs}
\usepgfplotslibrary{fillbetween}


\usepackage[version=4]{mhchem}
\usepackage{multido}

\definecolor{yorhabg}{HTML}{C8C2AA}
\definecolor{yorhafg}{HTML}{4D493E}
\definecolor{yorhagrid}{HTML}{B5AF9C}
\definecolor{gruvred}{HTML}{91221D}

\pagecolor{yorhabg}
\color{yorhafg}

% \import{project1 (pblock)}{preamble.sty}

% Removes padding above title
\usepackage{titling}
\setlength{\droptitle}{-10em}

% Font package
\usepackage[T1]{fontenc}

\usepackage{fouriernc}

\usepackage{calc}


\usepackage{sectsty}
\usepackage{graphicx}
\usepackage{amsmath}
\usepackage{amssymb}
\usepackage[most]{tcolorbox}

\usepackage{tikz}
\usepackage{eso-pic}
\usetikzlibrary{calc,shadows.blur}

\usepackage[backend=biber,style=ieee,citestyle=numeric-comp,sorting=none]{biblatex}
\addbibresource{sample.bib}

\AddToShipoutPictureBG{%
	\begin{tikzpicture}[remember picture, overlay,
		help lines/.append style={line width=0.05pt, color=yorhagrid}]
		\draw[help lines] (current page.south west) grid[step=5pt]
		(current page.north east);
	\end{tikzpicture}%
}

% Margins
\topmargin=0in
\evensidemargin=0in
\oddsidemargin=0in
\textwidth=6.5in
\textheight=9.0in
\headsep=0.25in

\AtBeginEnvironment{tcolorbox}{\small}

\newtcolorbox[auto counter, number within=section]{question}{%
	enhanced,
	colback=yorhabg,
	colframe=yorhafg,
	coltext=yorhafg,
	coltitle=yorhabg,
	title={\textbf{Question~\thetcbcounter}},
	arc=0pt,
	outer arc=0pt,
	drop shadow southeast,
	sharp corners,
}

\newtcolorbox{solution}{%
	enhanced,
	breakable,
	colback=yorhabg,
	colframe=yorhafg,
	coltext=yorhafg,
	coltitle=yorhabg,
	arc=0pt,
	outer arc=0pt,
	drop shadow southeast,
	sharp corners,
}

\newtcolorbox{imp}{enhanced,arc=0mm,colback=yorhabg,colframe=yorhafg,leftrule=10mm,coltext=yorhafg,%
	overlay={\node[anchor=west,outer sep=1pt] at (frame.west) {\scalebox{1}{\textcolor{yorhabg}{\textbf{Info}}}}; }}

\newcommand\bb[1]{\textcolor{yorhafg}{\textbf{#1}}}

\newcommand\ph{\textrm{pH}}
\newcommand\poh{\textrm{pOH}}

\newcommand\mybox[2][]{\tikz[overlay]\node[fill=yorhafg, color=yorhabg, inner sep=2pt, anchor=text, rectangle, rounded corners=1mm,#1] {#2};\phantom{#2}}


\pgfplotsset{
	every axis/.append style={
		legend style={
			fill=yorhagrid,     % dark background
			draw=yorhafg,        % white border
			text=yorhafg,
			rounded corners
		}
	}
}


\title{Math Prep Course Day One}
\author{Ashhad}

\begin{document}

\maketitle

\section{Graphing}

\begin{question}
	Find the area between the curves \(y^2=4ax\) and \(x^2 = 4ay\)
\end{question}

\begin{center}
	\begin{tikzpicture}
		\begin{axis}[xmin=-2, xmax=5,
			axis lines=middle]
			\addplot[name path=blue, color=blue!60!black, thick]{x^2/4};
			\addlegendentry{\(x^2 = 4ay\)}
			\addplot[name path=red, color=gruvred, thick, domain=0:5, samples=200]{sqrt(4*x)};
			\addplot[color=gruvred, thick, domain=0:5, samples=200]{-sqrt(4*x)};
			\addlegendentry{\(y^2 = 4ax\)}
			\addplot[yorhagrid] fill between[of=red and blue, soft clip = {domain = 0:4}];
		\end{axis}
	\end{tikzpicture}
\end{center}


First we must find the intersection points, we can obtain them as follows
\begin{align*}
	y^2 &= 4ax \\
	x &= \frac{y^2}{4a} \\
	x^2 &= 4ay \\
	\frac{y^4}{16a^2} &= 4ay \\
	\frac{y^4 - 64a^3y}{16a^2} &= 0 \\
	y(y^3 - 64a^3) &= 0 \\
\end{align*}
Resolving it further we get \(y=0,\ y=4a\). Resubstituting to find \(x\), we obtain \(x=0,\ x=4a\) as well.

Given that we only have two intersection points, we can simply compute the integral of the two functions over the interval \([0, 4a]\) and take the absolute value of the difference (since area will always be positive)

So integrating \(y^2 = 4ax\), note that we only need the positive right branch, thus our function becomes \(y = 2\sqrt{ax}\)
\begin{align*}
	F(x) &= \int_{0}^{4a} 2\sqrt{ax}dx \\
		&= 2\sqrt{a} \int_{0}^{4a} \sqrt{x} \\
		&= 2\sqrt{a} \left[\frac{2x^{\frac{3}{2}}}{3}\right]^{4a} _{0} \\
		&= \frac{4}{3}\sqrt{a} \ [8a^{\frac{3}{2}} - 0] \\
		&= \frac{32a^2}{3}
\end{align*}

Now we integrate \(x^2 = 4ay\).
\begin{align*}
	f(x) &= \frac{x^2}{4a} \\
	F(x) &= \frac{1}{4a} \int_0^{4a} x^2 \\
		&= \frac{1}{4a} \left[\frac{x^3}{3}\right]^{4a} _0 \\
		&= \frac{1}{12a} [64a^3 - 0] \\
		&= \frac{16a^2}{3}
\end{align*}

Thus the difference is \(\frac{16a^2}{3}\) and that is precisely the area between the two curves.

\begin{question}
	Sketch the curve \(\frac{y^2}{a^2} + \frac{x^2}{b^2} = 1\) and find the value for which the tangent is parallel to the y-axis
\end{question}

\begin{center}
	\begin{tikzpicture}
		\begin{axis}[axis lines = middle, xmin=-1, xmax=1, ymin=-1.5, ymax=1.5]
			\addplot[color=gruvred, domain=-0.70710678:0.70710678, samples=200, thick]{sqrt(1-2*x^2)};
			\addplot[color=gruvred, domain=-0.70710678:0.70710678, samples=200, thick]{-sqrt(1-2*x^2)};
			\addlegendentry{\(\frac{y^2}{a^2} + \frac{x^2}{b^2} = 1\)}
		\end{axis}
	\end{tikzpicture}
\end{center}

In order to find the tangents parallel to the y-axis, we must find where the derivative of the function equals (or approaches) \(\pm\infty\).

\begin{align*}
	a^2x^2 + b^2y^2 &= a^2b^2 \\
	2a^2x + 2b^2yy' &= 0 \\
	y' &= -\frac{2a^2x}{2b^2y} \\
\end{align*}

Thus that happens when \(y =0\). Plugging this into the equation of the ellipse to find the corresponding x values we obtain 

\begin{align*}
	0 + \frac{x^2}{b^2} &= 1 \\
	x^2 &= b^2 \\
	x &= \pm b \\
\end{align*}

The two tangent lines are therefore \(x = \pm b\).  

\begin{question}
	Draw the curve \(y^2 = a^2 + \sin^2 x\) where \(a^2 > 0\).
\end{question}

\begin{center}
	\begin{tikzpicture}
		\begin{axis}[axis lines = middle, ymax=2.5, ymin=-2.5]
			\addplot[color=gruvred, thick, samples = 200]{sqrt(1 + (sin(deg(x)))^2)};
			\addplot[color=gruvred, thick, samples = 200]{-sqrt(1 + (sin(deg(x)))^2)};
			\addlegendentry{\(y^2 = a^2 + \sin ^2 x\)}
		\end{axis}
	\end{tikzpicture}
\end{center}

\section{Indefinite Integrals}

\begin{question}
	Integrate
	\begin{align*}
		\frac{x}{4-x^2}
	\end{align*}
\end{question}

Using the substitution \(x = 2\sin \theta\) we obtain
\begin{align*}
	\int \frac{x}{4-x^2}dx &= \int \frac{2\sin\theta}{4-4\sin^2\theta}(2\cos\theta d\theta) \\
		&= \frac{4}{4} \frac{\sin\theta}{\cos^2\theta}{\cos\theta d\theta} \\
		&= \frac{\sin\theta}{\cos\theta}d\theta \\
		&= -\ln|\cos\theta| \\
		&= -\ln|\sqrt{1 - \sin^2\theta}| \\
		&= -\ln\left|\sqrt{1 - \frac{x^2}{4}}\right| \\
		&= -\frac{1}{2} \ln\left|1 - \frac{x^2}{4}\right| + C \\
\end{align*}

\begin{question}
	Integrate
\begin{align*}
	\frac{2x}{4-x^2}
\end{align*}
\end{question}

Using the last question,

\begin{align*}
	\int \frac{2x}{4-x^2} &= 2\int \frac{x}{4-x^2} \\
		&= -\ln\left|1-\frac{x^2}{4}\right| + C
\end{align*}

\begin{question}
	Integrate
\begin{align*}
	x \sin x
\end{align*}
\end{question}

We can solve this by using integration by parts

\begin{align*}
	\int x\sin x &= x \int \sin x\ dx - \int \int \sin x\  dx \\
		&= -x\cos x + \sin x + C\\
\end{align*}

\begin{question}
	Integrate
\begin{align*}
	e^{3x}x^2
\end{align*}
\end{question}

Again, this may be solved by repeated applications of IBP

\begin{align*}
	\int e^{3x}x^2 &= x^2 \int e^{3x} - \int \int e^{3x} 2x \\
		&= x^2 \frac{e^{3x}}{3} - \frac{2}{3}\int e^{3x} x \\
		&= x^2 \frac{e^{3x}}{3} - \frac{2}{3}\left[x \int e^{3x} - \int \int e^{3x}\right] \\
		&= x^2 \frac{e^{3x}}{3} - \frac{2}{3}\left[x \frac{e^{3x}}{3} - \frac{e^{3x}}{9}\right] \\
		&= \frac{e^{3x}}{3}\left[x^2 - \frac{2x}{3} + \frac{2}{9}\right] + C
\end{align*}

\begin{question}
	Integrate
\begin{align*}
	\frac{1}{(x-1)(x-3)(x+3)}
\end{align*}
\end{question}

To integrate this we first resolve by partial fractions
\begin{align*}
	\frac{1}{(x-1)(x-3)(x+3)} &= \frac{A}{x-1} + \frac{B}{x-3} + \frac{C}{x+3} \\
	1 &= A(x^2 - 9) +B(x^2 +2x -3) +C(x^2 -4x +3) \\
	2C &= B \\
	A + B + C &= 0 \\
	A + 3C &= 0 \\
	1 + 9A +3B - 3C &= 0 \\ 
	1 + 9A +3C &= 0 \\
	1 + 8A &= 0 \\
	A &= -\frac{1}{8} \\
	C &= \frac{1}{24} \\
	B &= \frac{1}{12}
\end{align*}

Now we can integrate this

\begin{align*}
	\int \frac{1}{(x-1)(x-3)(x+3)} &= -\frac{1}{8}\int \frac{1}{x-1} + \frac{1}{12}\int \frac{1}{x-3} + \frac{1}{24}\int \frac{1}{x+3}  \\
	&= -\frac{1}{8}\ln|x-1| +\frac{1}{12}\ln|x-3| + \frac{1}{24}\ln|x+3| \\
	&= \frac{1}{24}\left[-3\ln|x-1| +2\ln|x-3| +\ln|x+3|\right] + C
\end{align*}


\end{document}

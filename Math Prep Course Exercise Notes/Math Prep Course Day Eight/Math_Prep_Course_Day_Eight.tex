\documentclass[a4paper]{article}

\usepackage{import}
\usepackage{xcolor}
\usepackage{textcomp,mathcomp}
\usepackage{cancel}
\usepackage{float}
\usepackage{forloop}
\usepackage{pgfplots}
\usepackage{booktabs}
\usepgfplotslibrary{fillbetween}


\usepackage[version=4]{mhchem}
\usepackage{multido}

\definecolor{yorhabg}{HTML}{C8C2AA}
\definecolor{yorhafg}{HTML}{4D493E}
\definecolor{yorhagrid}{HTML}{B5AF9C}
\definecolor{gruvred}{HTML}{91221D}

\pagecolor{yorhabg}
\color{yorhafg}

% \import{project1 (pblock)}{preamble.sty}

% Removes padding above title
\usepackage{titling}
\setlength{\droptitle}{-10em}

% Font package
\usepackage[T1]{fontenc}

\usepackage{fouriernc}

\usepackage{calc}


\usepackage{sectsty}
\usepackage{graphicx}
\usepackage{amsmath}
\usepackage{amssymb}
\usepackage[most]{tcolorbox}

\usepackage{tikz}
\usepackage{eso-pic}
\usetikzlibrary{calc,shadows.blur}

\usepackage[backend=biber,style=ieee,citestyle=numeric-comp,sorting=none]{biblatex}
\addbibresource{sample.bib}

\AddToShipoutPictureBG{%
	\begin{tikzpicture}[remember picture, overlay,
		help lines/.append style={line width=0.05pt, color=yorhagrid}]
		\draw[help lines] (current page.south west) grid[step=5pt]
		(current page.north east);
	\end{tikzpicture}%
}

% Margins
\topmargin=0in
\evensidemargin=0in
\oddsidemargin=0in
\textwidth=6.5in
\textheight=9.0in
\headsep=0.25in

\AtBeginEnvironment{tcolorbox}{\small}

\newtcolorbox[auto counter, number within=section]{question}{%
	enhanced,
	colback=yorhabg,
	colframe=yorhafg,
	coltext=yorhafg,
	coltitle=yorhabg,
	title={\textbf{Question~\thetcbcounter}},
	arc=0pt,
	outer arc=0pt,
	drop shadow southeast,
	sharp corners,
}

\newtcolorbox{solution}{%
	enhanced,
	breakable,
	colback=yorhabg,
	colframe=yorhafg,
	coltext=yorhafg,
	coltitle=yorhabg,
	arc=0pt,
	outer arc=0pt,
	drop shadow southeast,
	sharp corners,
}

\newtcolorbox{imp}{enhanced,arc=0mm,colback=yorhabg,colframe=yorhafg,leftrule=10mm,coltext=yorhafg,%
	overlay={\node[anchor=west,outer sep=1pt] at (frame.west) {\scalebox{1}{\textcolor{yorhabg}{\textbf{Info}}}}; }}

\newcommand\bb[1]{\textcolor{yorhafg}{\textbf{#1}}}

\newcommand\ph{\textrm{pH}}
\newcommand\poh{\textrm{pOH}}

\newcommand\mybox[2][]{\tikz[overlay]\node[fill=yorhafg, color=yorhabg, inner sep=2pt, anchor=text, rectangle, rounded corners=1mm,#1] {#2};\phantom{#2}}


\pgfplotsset{
	every axis/.append style={
		legend style={
			fill=yorhagrid,     % dark background
			draw=yorhafg,        % white border
			text=yorhafg,
			rounded corners
		}
	}
}


\title{Math Prep Course Day One}
\author{Ashhad}

\begin{document}

\maketitle

\section{Definite Integrals}

\begin{question}
	Find \(\int_{-3}^{3} f(x)\ dx\) Where 
\begin{align*}
	f(x) = 
	\begin{cases}
		-x & x \in [-3, -1] \\
		1 & x \in [-1, 1] \\
		x & x \in [1, 3] 
	\end{cases}
\end{align*} 
\end{question}

To evaluate this, split the integral into three parts
\begin{align*}
	\int_{-3}^{3} f(x)\ dx &= \int_{-3}^{-1} -x\ dx + \int_{-1}^{1} dx + \int_{1}^{3} x\ dx \\
		&= \int_{1}^{3} x\ dx + 2 + \int_{1}^{3} x\ dx \\
		&= 2\left[\frac{x^2}{2}\right]_{1}^{3} + 2 \\
		&= 2\left[\frac{9}{2} - \frac{1}{2}\right] + 2 \\
		&= 10 \\
\end{align*}

\begin{question}
	Calculate \(\int_{0}^{2\pi} \sin^2 x\ dx\)
\end{question}

\begin{align*}
	\int_{0}^{2\pi} \sin^2 x\ dx &= \frac{1}{2} \int_{0}^{2\pi} 1 - \cos (2x)\ dx \\
	&= \frac{1}{2} \left[x -\sin (2x)\right]_{0}^{2\pi} \\
	&= \frac{1}{2} [2\pi] \\
	&= \pi
\end{align*}

\begin{question}
	Calculate \(\int_{-\frac{\pi}{2}}^{\frac{\pi}{2}} \sin^3 x\ dx\)
\end{question}

Since we have \(\sin^3 (-x) = -\sin^3 x\), we may write
\begin{align*}
	\int_{-\frac{\pi}{2}}^{\frac{\pi}{2}} \sin^3 x\ dx &= \int_{-\frac{\pi}{2}}^{0} \sin^3 x\ dx + \int_{0}^{\frac{\pi}{2}} \sin^3 x\ dx \\
	&= \int_{0}^{\frac{\pi}{2}} \sin^3 (-x)\ dx + \int_{0}^{\frac{\pi}{2}} \sin^3 x\ dx \\
	&= \int_{0}^{\frac{\pi}{2}} \sin^3 x + \sin^3 (-x) \\
	&= 0
\end{align*}

\begin{question}
	Calculate \(\int_{0}^{\infty} xe^{-x}\ dx\)
\end{question}

Using Integration by parts
\begin{align*}
	\int_{0}^{\infty} xe^{-x}\ dx &= \left[x \int e^{-x}\ dx\right]_{0}^{\infty} - \int_{0}^{\infty} \int e^{-x}\ dx \\
	&= \left[-xe^{-x}\right]_{0}^{\infty} + \int_{0}^{\infty} e^{-x} \\
	&= [0 - 0] + [1 - 0] \\
	&= 1 
\end{align*}

\begin{question}
	If \(f(t+a) = f(t)\) then find \(\int_{0}^{na} f(t)\ dt\) where \(\int_{0}^{a} f(t)\ dt = \frac{a}{2}\)
\end{question}

For this, we split the integral into n integrals.

\begin{align*}
	\int_{0}^{na} f(t)\ dt &= \sum_{i=0}^{n-1} \int_{ia}^{(i+1)a} f(t)\ dt \\
\end{align*}

On each integral \(I_i\) in the sum, substitute in the expression \(t = t - ia\)

\begin{align*}
	\sum_{i=0}^{n-1} \int_{ia}^{(i+1)a} f(t)\ dt &= \sum_{i=0}^{n-1} \int_{0}^{a} f(t-ia)\ dt \\
	 &= \sum_{i=0}^{n-1} \int_{0}^{a} f(t)\ dt \\
	 &= \sum_{i=0}^{n-1} \frac{a}{2} \\
	 &= \frac{na}{2}
\end{align*}

\begin{question}
	Show that
	\begin{align*}
		\left[
			\begin{matrix}
				\cos\theta & \sin\theta \\
				-\sin\theta & \cos\theta \\
			\end{matrix}
		\right]
	\end{align*}
	Is orthogonal
\end{question}

An orthogonal matrix is one whose rows/columns form an orthonormal set.

Let \(e_1\) represent the first column and \(e_2\) represent the second one. To show orthogonality, it is sufficient to check that the dot product of \(e_1\) and \(e_2\) is zero.

\begin{align*}
	e_1 \cdot e_2 &= (\cos\theta - \sin\theta) \cdot (\sin\theta + \cos\theta) \\
	&= (\cos\theta\sin\theta - \sin\theta\cos\theta) \\
	&= 0
\end{align*}

Next we must show that they are both also unit vectors

\begin{align*}
	||e_1|| &= (\cos\theta)^2 + (-\sin\theta)^2 = 1 \\
	||e_2|| &= (\sin\theta)^2 + (\cos\theta)^2 = 1 \\
\end{align*}

Therefore the matrix is orthogonal.

\begin{question}
	Find the inverse of 
	\begin{align*}
		B =
		\left[
		\begin{matrix}
			-3 & 1 & 1 \\
			2 & 3 & -1 \\
			4 & 2 & 1 \\
		\end{matrix}
		\right]
	\end{align*}
\end{question}

Row Operations:
\begin{enumerate}
	\item \(R_1 \leftarrow R_1/(-3)\)
	\item \(R_2 \leftarrow R_2 - 2R_1\)
	\item \(R_3 \leftarrow R_3 - 4R_1 \leftarrow\) Matrix 2 
	\item \(R_2 \leftarrow R_2/(\frac{11}{3})\)
	\item \(R_1 \leftarrow R_1 + \frac{1}{3}R_2\)
	\item \(R_3 \leftarrow R_3 - \frac{10}{3}R_2\leftarrow\) Matrix 3 
	\item \(R_3 \leftarrow R_3/(\frac{29}{11})\)
	\item \(R_1 \leftarrow R_1 + \frac{4}{11}R_3\)
	\item \(R_2 \leftarrow R_2 - \frac{1}{11}R_3\leftarrow\) Matrix 4
\end{enumerate}

\begin{align}
	\left[	
	\begin{array}{ccc|ccc}
		-3 & 1 & 1 & 1 & 0 & 0 \\
		2 & 3 & -1 & 0 & 1 & 0 \\
		4 & 2 & 1 & 0 & 0 & 1 \\
	\end{array}
	\right] \\
	\left[	
	\begin{array}{ccc|ccc}
		1 & -\frac{1}{3} & -\frac{1}{3} & -\frac{1}{3} & 0 & 0 \\
		0 & \frac{11}{3} & -\frac{1}{3} & \frac{2}{3} & 1 & 0 \\
		0 & \frac{10}{3} & \frac{7}{3} & \frac{4}{3} & 0 & 1 \\
	\end{array}
	\right] \\
	\left[	
	\begin{array}{ccc|ccc}
		1 & 0 & -\frac{12}{33} & -\frac{9}{33} & \frac{1}{11} & 0 \\
		0 & 1 & -\frac{1}{11} & \frac{2}{11} & \frac{3}{11} & 0 \\
		0 & 0 & \frac{29}{11} & \frac{8}{11} & -\frac{10}{11} & 1 \\
	\end{array}
	\right] \\
	\left[	
	\begin{array}{ccc|ccc}
		1 & 0 & 0 & -\frac{5}{29} & -\frac{1}{29} & \frac{4}{29} \\
		0 & 1 & 0 & \frac{6}{29} & \frac{7}{29} & \frac{1}{29} \\
		0 & 0 & 1 & \frac{8}{29} & -\frac{10}{29} & \frac{11}{29} \\
	\end{array}
	\right]
\end{align}

So the inverse is
\begin{align*}
	\frac{1}{29} \left[
	\begin{matrix}
		-5 & -1 & 4 \\
		6 & 7 & 1\\
		8 & -10 & 11\\
	\end{matrix}
	\right]
\end{align*}

\end{document}

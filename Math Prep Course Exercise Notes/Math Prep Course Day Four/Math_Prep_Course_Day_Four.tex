\documentclass[a4paper]{article}

\usepackage{amsthm}
\usepackage{amssymb}
\usepackage{amsmath}
\usepackage{xcolor}
\usepackage{pgfplots}
\usepackage{booktabs}
\usepackage[most]{tcolorbox}
\usepgfplotslibrary{fillbetween}

\newtheorem{remark}{Remark}[section]
\newtheorem{definition}{Definition}[section]
\theoremstyle{definition}
\newtheorem{lemma}{Lemma}[section]
\newtheorem{theorem}{Theorem}[section]
\newtheorem{example}{Example}[section]
\newtheorem{corollary}{Corollary}[section]
\newtheorem{condition}{Assumption}[section]
\newtheorem{insight}{Insight}[section]

\newtheorem{problem}{Problem}[section]



\newtheorem*{solution}{Solution}

\setlength{\parindent}{0pt}   % no paragraph indentation
\setlength{\parskip}{20pt}    % vertical space between paragraphs

\title{Math Prep Course Day Four}
\author{Ashhad Shahzad}
\date{September 2025}

\begin{document}

\maketitle

\section{Function Composition}

\begin{problem}{Find The Domain of}
\begin{align*}
f(x) = \frac{7 -\sqrt{x^2-9}}{\sqrt{25-x^2}}
\end{align*}
\end{problem}

\begin{solution}
We can decompose \(f(x)\) into its numerator and denominator functions and take the intersection of their domain. That can be done as follows
\begin{align*}
x^2 - 9 &\geq 0 \\
x^2 &\geq 9 \\
x &\geq |3| \\ \\
25 - x^2 &\geq 0 \\
x^2 &\leq 25 \\
x &\leq |5|
\end{align*}

And so the intersection of the domains is simply \((-5, -3] \cup [3, 5)\).
\end{solution}


\begin{problem}
Let \(f(x) = \sqrt{x}\), \(g(x) = \frac{4}{5-x}\), \(h(x) = x^2\). Find
\begin{align*}
(h(h \circ g\circ f - f))(4)
\end{align*}
And it's domain
\end{problem}

\begin{solution}
\begin{align*}
h(h \circ g \circ f- f) &= \left(\left(\frac{4}{5-\sqrt{x}}\right)^2 -\sqrt{x}\right)^2 \\
(h(h \circ g\circ f -f))(4) &= \left(\left(\frac{4}{5-\sqrt{4}}\right)^2 - \sqrt{4}\right)^2 \\
    &= \left(\left(\frac{4}{5-2}\right)^2 - 2\right)^2 \\
    &= \left(\left(\frac{4}{3}\right)^2 - 2\right)^2 \\
    &= \frac{4}{81}
\end{align*}
The domain is clearly \([0, \infty)/\{25\}\)
\end{solution}

\section{Polynomials}
I've solved some of these using long division instead of the remainder theorem for my own sake. The remainder theorem is however far more efficient.
\begin{problem}
Given that \((x-3)\) is a factor of 
\begin{align*}
f(x) = x^3 - 2x^2 + kx +6
\end{align*}
Show that \(k=-5\).
\end{problem}
\begin{solution}
\[
\begin{array}{r|l}
& x-3 \\
\midrule
x^2 & x^3 -2x^2 +kx +6 \\
& -x^3 +3x^2 \\
\midrule
x & x^2 +kx +6 \\
& -x^2 + 3x \\
\midrule
k+3 & (k+3)x +6 \\
& -(k+3)x +3k+39\\
\midrule
& 3k+15
\end{array}
\]

Since \(3k+15 = 0\), we obtain \(k=-5\)
\end{solution}

\begin{problem}
Evaluate
\begin{align*}
(x^5 -4x^4 +2x^3 -3x^2 +4x +1) \div (x^2 +x +1)
\end{align*}
\end{problem}

\begin{solution}
\[
\begin{array}{r|l}
& (x^2 + x + 1) \\
\midrule
x^3 & x^5- 4x^4 + 2x^3 -3x^2 +4x +1   \\
& -x^5 -x^4 -x^3 \\
\midrule
-5x^2 & -5x^4 +x^3 -3x^3 +4x +1 \\
&5x^4 + 5x^3 + 5x^2 \\
\midrule
6x & 6x^3 +2x^2 +4x +1 \\
& -6x^3 -6x^2 -6x \\
\midrule
-4 & -4x^2 -2x +1 \\
& 4x^2 +4x +4 \\
\midrule
& 2x + 5
\end{array}
\]
\end{solution}

\begin{problem}
Find the remainder
\begin{align*}
(4x^3 -2x^2 +x +1) \div  (x-1)
\end{align*}
\end{problem}

\begin{solution}
\[
\begin{array}{r|l}
& x-1 \\
\midrule
4x^2  & 4x^3 -2x^2 +x +1 \\
& -4x^3 + 4x^2 \\
\midrule
2x & 2x^2 +x +1 \\
& -2x^2 + 2x \\
\midrule
3 & 3x+1 \\
& -3x + 3 \\
\midrule
&4
\end{array}
\]
\end{solution}


\begin{problem}
\begin{align*}
f(x) = ax^3 -7x^2 +1
\end{align*}
Has the factor \((x-1)\). Find \(a\).
\end{problem}

\begin{solution}
\[
\begin{array}{r|l}
& x-1 \\
\midrule
ax^2 & ax^3 - 7x^2 +1 \\
& -ax^3 + ax^2 \\
\midrule
(a-7)x & (a-7)x^2 +1 \\
& (7-a)x^2 + (a-7)x \\
\midrule
(7-a) & (a-7)x +1 \\
& (7-a)x + (a-7) \\
\midrule
& a -6
\end{array}
\]

Since the remainder must be zero, we obtain \(a=6\)
\end{solution}

\begin{problem}
\begin{align*}
5x^6 -3x^5 -x^4 +1  = (x-1)(x-2) - x^2 +3x -1
\end{align*}
\end{problem}

\begin{solution}
\begin{align*}
5x^6 -3x^5 -x^4 + 1 &= x^2 -3x +2 -x^2 +3x -1 \\
5x^6 -3x^5 -x^4 &= 0 \\
x^4(5x^2 -3x -1) &= 0 \\
\end{align*}
Solving for the quadratic, we obtain the solutions 
\[
x= \left\{0,\ \frac{3 \pm \sqrt{29}}{10} \right\}
\]
\end{solution}

\section{Logarithmic and Exponential Functions}

\begin{problem}
\begin{align*}
\log_2 x  = \log_2 x^2 -4
\end{align*}
\end{problem}

\begin{solution}
\begin{align*}
\log_2 x &= 2 \log_2 x - 4 \\
\log_2 x &= 4 \\
x &= 2^4 \\
x &= 16
\end{align*}
\end{solution}

\begin{problem}
\begin{align*}
2^x = 64
\end{align*}
\end{problem}

\begin{solution}
\begin{align*}
2^x &= 64 \\
x &= \log_2(2^6) \\
x &= 6 \\
\end{align*}
\end{solution}

\begin{problem}
\begin{align*}
\ln(x-1) - \ln(x^2-1) = e
\end{align*}
\end{problem}

\begin{solution}
\begin{align*}
\ln\left(\frac{x-1}{x^2-1}\right) &= e \\
\ln(x+1) &= -e \\
x+1 &= e^{-e} \\
x &= e^{-e} -1
\end{align*}
However, \(e^{-e} -1 < 1\), which makes the expression \(\ln(x-1)\) invalid, therefore we have an extraneous solution and the equation is unsatisfiable in \(\mathbb{R}\)
\end{solution}

\begin{problem}
Find the domain and range of \(x = \ln(e^x)\).
\end{problem}

\begin{solution}
Observe that \(e^x\) is defined over all of \(\mathbb{R}\) and has range \((0, \infty)\). Note that this is precisely the domain of the \(\ln\) function, which has the range \((-\infty, \infty)\). Therefore the domain and range of the function are both \(\mathbb{R}\).

In fact its just the constant function \(x\) since its the composition of a function with its inverse lmao.
\end{solution}

\begin{problem}
\begin{align*}
\log_2 x + \log_x 2 +1 = 0
\end{align*}
\end{problem}

\begin{solution}
\begin{align*}
\log_2 x + \frac{\log_2 2}{\log_2 x} +1 &=0 && \because \log_ab = \frac{\log_c b}{\log_c a} \\
\log_2 x + \frac{1}{\log_2x} +1 &= 0 \\
(\log_2x)^2 + \log_2 x+ 1 &= 0 \\
y^2 + y + 1 &=0 && \because let\ y =\log_2 x \\
\end{align*}

Since the solution is complex, I'm not going to evaluate it further.
\end{solution}

\begin{problem}
\begin{align*}
\log_5 x+ \log_{10}8 = 1
\end{align*}
\end{problem}

\begin{solution}
\begin{align*}
\log_5x &= 1- \log_{10}8 \\
x &= 5^{(1-\log_{10} 8)}
\end{align*}
\end{solution}

\begin{problem}
\begin{align*}
(e^x)^2 + \ln e^{e^x} + e^{\ln_e 5} = 0
\end{align*}
\end{problem}

\begin{solution}
\begin{align*}
(e^x)^2 + e^x + 5 &= 0 \\
y^2 + y + 5 &= 0 && \because y = e^x
\end{align*}

Again, the solution is complex.
\end{solution}

\begin{problem}
\begin{align*}
x^2 = e^{\ln\left(\frac{1}{5}x\right)+ \ln(5x)} + \log_{10}e^{10}
\end{align*}
\end{problem}

\begin{solution}
\begin{align*}
x^2 &= \left(\frac{x}{5}\right)(5x) + 10 \log_{10} e \\
0 &= 10\log_{10} {e}
\end{align*}

Thus the equation has no solutions.
\end{solution}

\end{document}

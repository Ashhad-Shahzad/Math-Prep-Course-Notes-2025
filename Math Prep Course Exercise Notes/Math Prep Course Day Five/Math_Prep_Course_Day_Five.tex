\documentclass{article}
\usepackage{import}
\usepackage{xcolor}
\usepackage{textcomp,mathcomp}
\usepackage{cancel}
\usepackage{float}
\usepackage{forloop}
\usepackage{pgfplots}
\usepackage{booktabs}
\usepackage{nicematrix}
\usepgfplotslibrary{fillbetween}


\usepackage[version=4]{mhchem}
\usepackage{multido}

\definecolor{yorhabg}{HTML}{C8C2AA}
\definecolor{yorhafg}{HTML}{4D493E}
\definecolor{yorhagrid}{HTML}{B5AF9C}
\definecolor{gruvred}{HTML}{91221D}

\pagecolor{yorhabg}
\color{yorhafg}

% \import{project1 (pblock)}{preamble.sty}

% Removes padding above title
\usepackage{titling}
\setlength{\droptitle}{-10em}

% Font package
\usepackage[T1]{fontenc}

\usepackage{fouriernc}

\usepackage{calc}


\usepackage{sectsty}
\usepackage{graphicx}
\usepackage{amsmath}
\usepackage{amssymb}
\usepackage[most]{tcolorbox}

\usepackage{tikz}
\usepackage{eso-pic}
\usetikzlibrary{calc,shadows.blur}

\usepackage[backend=biber,style=ieee,citestyle=numeric-comp,sorting=none]{biblatex}
\addbibresource{sample.bib}

\AddToShipoutPictureBG{%
	\begin{tikzpicture}[remember picture, overlay,
		help lines/.append style={line width=0.05pt, color=yorhagrid}]
		\draw[help lines] (current page.south west) grid[step=5pt]
		(current page.north east);
	\end{tikzpicture}%
}

% Margins
\topmargin=0in
\evensidemargin=0in
\oddsidemargin=0in
\textwidth=6.5in
\textheight=9.0in
\headsep=0.25in

\AtBeginEnvironment{tcolorbox}{\small}

\newtcolorbox[auto counter, number within=section]{question}{%
	enhanced,
	colback=yorhabg,
	colframe=yorhafg,
	coltext=yorhafg,
	coltitle=yorhabg,
	title={\textbf{Question~\thetcbcounter}},
	arc=0pt,
	outer arc=0pt,
	drop shadow southeast,
	sharp corners,
}

\newtcolorbox{boxx}{%
	enhanced,
	breakable,
	colback=yorhabg,
	colframe=yorhafg,
	coltext=yorhafg,
	coltitle=yorhabg,
	arc=0pt,
	outer arc=0pt,
	drop shadow southeast,
	sharp corners,
}

\newtcolorbox{imp}{enhanced,arc=0mm,colback=yorhabg,colframe=yorhafg,leftrule=10mm,coltext=yorhafg,%
	overlay={\node[anchor=west,outer sep=1pt] at (frame.west) {\scalebox{1}{\textcolor{yorhabg}{\textbf{Info}}}}; }}

\newcommand\bb[1]{\textcolor{yorhafg}{\textbf{#1}}}

\newcommand\ph{\textrm{pH}}
\newcommand\poh{\textrm{pOH}}

\newcommand\mybox[2][]{\tikz[overlay]\node[fill=yorhafg, color=yorhabg, inner sep=2pt, anchor=text, rectangle, rounded corners=1mm,#1] {#2};\phantom{#2}}


\pgfplotsset{
	every axis/.append style={
		legend style={
			fill=yorhagrid,     % dark background
			draw=yorhafg,        % white border
			text=yorhafg,
			rounded corners
		}
	}
}


\title{Math Prep Course Day Five}
\author{Ashhad Shahzad}
\date{September 2025}

\begin{document}

\maketitle

\section{Trigonometric funtions}

\begin{question}{}{}
Find the Maximum and minimum value of \(\sin^2(\theta) + \cos^4(\theta) \)
\end{question}

\begin{align*}
	f(\theta) &=\sin^2 (\theta) + (\cos^2 (1-\sin^2)) \theta \\
	f(\theta) &=\sin^2 (\theta) + \cos^2 (\theta) - \cos^2 (\theta) \sin^2 (\theta) \\
	f(\theta) &=1 - (\cos *\sin)^2 (\theta)
\end{align*}

Since the term \((\cos *\sin)^2\) is always positive, our maximum point will be when that term is equal to zero, which gives us a maximum value of \(1\). Similarly, our minimum point will be when that term is maximum. For that, we may solve

\begin{align*}
	\cos *\sin (\theta) &= \frac{\sin (2\theta)}{2} \\
\end{align*}

Since the maximum value of \(\sin\) is 1, the maximum value of \((\cos *\sin)^2\) turns out to be \(\frac{1}{4}\). Thus the minimum value of our function is \(\frac{3}{4}\)

\begin{question}
	If \(x+y+z =xyz\) then prove that
	\begin{align*}
		\frac{x}{1-x^2} + \frac{y}{1-y^2} + \frac{z}{1-z^2} = \frac{4xyz}{(1-x^2)(1-y^2)(1-z^2)}
	\end{align*}
\end{question}
I'm going to leverage a decent amount of shortcuts here because the calculation is so tedious if you write everything out. The notation \(\sum_{cyc}\) means that you will write out three terms alternating between the variables, i.e. \(\sum_{cyc}x+y\) means \((x+y) + (y+z) + (z+x)\). I highly recommend that if you aren't familiar with this notation that you explicitly write everything out and see how the steps follow, its a good exercise in notation.
\begin{align*}
	S_1 &= x+y+z \\
	S_2 &= xy +yz +zx \\
	S_3 &= xyz
\end{align*}
Now do the cross multiplication of the LHS with respect to each denominator.
Then our numerator becomes
\begin{align*}
	\sum_{cyc} x(1-y^2-z^2+y^2z^2) &= \sum_{cyc}x + \sum_{cyc}xy^2z^2 - \sum_{cyc}x(y^2+z^2) \\
	&= S_1 + S_3\sum_{cyc}xy - \sum_{cyc}x^2(y+z) \\
	&= S_3 + S_1\sum_{cyc}xy - \sum_{cyz}x^2(y+z) \\
	&= S_3 + \sum_{cyc}xyz + \sum_{cyc}(x^2y + y^2x) - \sum_{cyc}x^2(y+z)\\
	&= 4S_3
\end{align*}
The final solution thus becomes, as desired
\begin{align*}
	\frac{4xyz}{(1-x^2)(1-y^2)(1-z^2)}
\end{align*}

\begin{question}
	Find the number of solutions of
	\begin{align*}
		3\sin^2x - 7\sin x +2 = 0 && x\in [0, 5\pi]
	\end{align*}
\end{question}

Let \(\sin x = y\) and solve the quadratic 
\begin{align*}
	3y^2 -7y +2 &= 0 \\
\end{align*}
The solution set is \(\left\{\frac{1}{3}, 2\right\}\). Note that \(2\) is extraneous since \(\sin\) can never be greater than one. Note that in its cycle, from \(\left(0, \frac{\pi}{2}\right)\) sin increases from 0 to 1, by IVT it must pass through \(\frac{1}{3}\) and by monotonicity it does so only once. A similar conclusion can be drawn for \(\left(\frac{\pi}{2}, \pi\right)\). In the latter half, the \(\sin\) function becomes negative. Thus in \(5\pi\), the \(\sin\) function will hit \(\frac{1}{3}\) six times owing to its \(2\pi\) periodicity. 
\begin{question}
	Find the number of solutions of
	\begin{align*}
		\tan x + \sec x = 2\cos x && x \in [0, 2\pi]
	\end{align*}
\end{question}

\begin{align*}
	\frac{\sin x}{\cos x} + \frac{1}{\cos x} &= 2\cos x \\
	\frac{\sin x}{\cos^2 x} + \frac{1}{\cos ^2 x} - 2 &= 0 \\
	\sin x + 1 -2\cos^2 x &= 0 \\
	2\sin^2 + \sin x - 1 &= 0 \\
	2y^2 + y -1 &= 0
\end{align*}
This quadratic has the solutions \(\left\{-1, \frac{1}{2}\right\}\). In terms of angles this gives us \(\left\{\frac{3\pi}{2}, \frac{\pi}{6}, \frac{5\pi}{6}\right\}\). Note however that the first root in the set is extraneous since \(\cos\) is zero at that value making the expression undefined. The two solutions are therefore

\begin{align*}
	\left\{\frac{\pi}{6}, \frac{5\pi}{6}\right\} \\
\end{align*}

\begin{question}
	Solve \(\tan^2 x- (1-\sqrt{3})\tan x + \sqrt{3} < 0\)
\end{question}

\begin{align*}
	y^2 - (1 - \sqrt{3})y + \sqrt{3} &< 0 \\
	y &= \frac{1 - \sqrt{3} \pm \sqrt{(4-2\sqrt{3}) -4\sqrt{3}}}{2} \\
	y &= \frac{1 - \sqrt{3} \pm \sqrt{24 - 16\sqrt{3}}}{2}
\end{align*}
Since the solutions are complex and the parabola is upward facing as the leading square coefficient is positive, the parabola lies completely above \(y = 0\) and thus the inequality is never satisfied. The solution is just the set \(\emptyset\)

\begin{question}
	Draw \(y = 5\sin \left(2x + \frac{\pi}{2}\right)\)
\end{question}

\begin{center}
	\begin{tikzpicture}
		\begin{axis}[
			axis lines = middle,
			domain=0:2*pi,
			samples=200,
			xtick={0, pi/4, pi/2, 3*pi/4, pi, 5*pi/4, 3*pi/2,7*pi/4, 2*pi},
			xticklabels={$0$, $\tfrac{\pi}{4}$, $\tfrac{\pi}{2}$, $\tfrac{3\pi}{4}$, $\pi$, $\tfrac{5\pi}{4}$, $\tfrac{3\pi}{2}$, $\tfrac{7\pi}{4}$, $2\pi$},
			ytick={-5,-2.5,0,2.5,5},
			xlabel={$x$},
			ylabel={$y$},
			grid=both
			]
			\addplot[color=gruvred, domain=0:2*pi, samples=200]
			{5*sin(deg(2*x + pi/2))};
			\addlegendentry{\(5\sin \left(2x + \frac{\pi}{2}\right)\)}
		\end{axis}
	\end{tikzpicture}
\end{center}
\end{document}
